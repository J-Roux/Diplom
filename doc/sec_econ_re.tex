\newcommand{\byr}{ руб.}

\section{ЭКОНОМИЧЕСКОЕ ОБОСНОВАНИЕ РАЗРАБОТКИ МОДУЛЯ ВЫДЕЛЕНИЯ ИНФОРМАЦИОННЫХ ОБРАЗОВ ИЗ МУЗЫКАЛЬНОГО ПРОИЗВЕДЕНИЯ}
\label{sec:econ}

\subsection{Описание функций, назначения и потенциальных пользователей ПО}
\label{sub:econ:overview_appointment}
Модуль выделения информационных признаков из музыкального произведения предназначен для получения спектральных, временных и спектрально-временных  признаков из музыкального произведения. 
Модуль может быть использован в сервисах связанных с музыкой. Также может быть использован аналитиками данных в музыкальной сфере. 
Программный продукт относится к программному обеспечению по индивидуальному заказу для использования внутри организации-заказчика.

\subsection{Расчёт затрат на разработку ПО}
\label{sub:econ:expenses}
Для осуществления упрощённого расчёта затрат на разработку ПО следует произвести расчёт следующих статей:
\begin{itemize}
\item затраты на основную заработную плату разработчиков;
\item затраты на дополнительную заработную плату разработчиков;
\item отчисления на социальные нужды;
\item прочие затраты (амортизация оборудования, расходы на электроэнергию, командировочные расходы, накладные расходы и т.п.).
\end{itemize}

Расчёт затрат на основную заработную плату разработчиков \ref{table:econ:initial_data} осуществляется на численности и состава команды, размеров месячной заработной платы каждого из участников команды, а также общей трудоёмкости разработки программного обеспечения.

Основная заработная плата исполнителей на конкретное программное средство определяется по формуле:

\begin{equation}
  \label{eq:econ:total_salary}
  \text{З}_{\text{о}} = \sum^{n}_{i = 1}
                        \text{Т}_{\text{чi}} \cdot
                        \text{t}_{\text{i}}
                        \text{\,,}
\end{equation}
\begin{explanation}
где & $ \text{n} $ & количество исполнителей, занятых разработкой конкретного ПО; \\
    & $ \text{Т}_{\text{чi}} $ & часовая тарифная ставка \mbox{$ i $-го} исполнителя, руб; \\
    & $ \text{t}_{\text{i}} $ & трудоемкость работ, выполняемых \mbox{$ i $-го} исполнителем, час.
\end{explanation}

В проекте занято три человека: руководитель, разработчик и специалист по анализу данных.

Часовая заработная плата определяется на основе месячной заработной планы путём деления на количество рабочих часов в месяце (примем 166 часов). Для руководителя, при заработной плате равной 600 рублей, часовая заработная плата равна 3,61 рубля. Для разработчика, при заработной плате равно 500 рублей, часовая заработная плата равна 3,01 рубля. Для специалиста по анализу данных при зарплате 14700 рублей, часовая заработная плата равна 88,55 рубля
Трудоемкость определяется исходя из сложности разработки программного продукта и объема выполняемых им функций. В нашем случае она составляет 90 дней или 720 часов.
Тогда основная зарплата исполнителей равна:

\begin{equation}
  \label{eq:econ:total_salary_calc}
  \text{З}_{\text{о}} = (\num{3,01} + \num{3,61} + \num{88,55}) \times \num{720} = \num{68522,4} {\text{\byr}}
\end{equation}

\begin{table}[!ht]
\caption{Расчет затрат на основную заработную плату команды}
\label{table:econ:initial_data}
  \centering
  \begin{tabular}{| >{\raggedright}m{0.02\textwidth}
                  | >{\centering}m{0.17\textwidth}
                  | >{\centering}m{0.18\textwidth}
                  | >{\centering}m{0.12\textwidth}
                  | >{\centering}m{0.11\textwidth}
                  | >{\centering}m{0.11\textwidth}
                  | >{\centering\arraybackslash}m{0.12\textwidth}|}
    \hline
    {\begin{center}
    №
    \end{center} } & Участник команды & Выполняемые работы & Месячная заработная плата, р & Часовая заработная плата, р. & Трудоем-кость работ, ч. & Основная заработная плата, р. \\
    \hline
    1 & Руководитель проекта & Контроль, помощь & \num{600} & \num{3,61} & \num{720} & \num{2599,2} \\

    \hline
    2 & Программист 1-й категории & Разработка & \num{500} & \num{3,01} & \num{720} & \num{2167,2} \\

    \hline
    3 & Специалист по анализу данных & Разработка & \num{14700} & \num{88,55} & \num{720} & \num{63756} \\

    \hline
    \multicolumn{6}{|c|}{ПРЕМИЯ (50\%)} & \num{34261.2} \\

    \hline

    \multicolumn{6}{|c|}{Итого затраты на основную заработную плату разработчиков} & \num{102783,6}\\

    \hline

  \end{tabular}
\end{table}

Затраты на дополнительную заработную плату команды разработчиков включает выплаты, предусмотренные законодательством о труде (оплата отпусков, льготных часов, времени выполнения государственных обязанностей и других выплат, не связанных с основной деятельностью исполнителей), и определяется по формуле:

\begin{equation}
  \label{eq:econ:additional_salary}
  \text{З}_{\text{д}} =
    \frac {\text{З}_{\text{о}} \cdot \text{Н}_{\text{д}}}
          {\num{100}} \text{\,,}
\end{equation}
\begin{explanation}
    где & $ \text{З}_{\text{о}} $ & затраты на основную заработную плату с учетом премии, руб; \\
        & $ \text{Н}_{\text{д}} $ & норматив дополнительной заработной платы, \num{15} \%.
\end{explanation}

В результате подстановки получим:

\begin{equation}
  \label{eq:econ:additional_salary_calc}
  \text{З}_{\text{д}} =
    \frac{\num{102783,6} \times \num{15}}
         {\num{100}} = \num{15417,52}{\text{\byr}}
\end{equation}

Отчисления на социальные нужды (в фонд социальной защиты населения и на обязательное страхование) определяются в соответствии с действующими законодательными актами по формуле:

\begin{equation}
  \label{eq:econ:com}
  \text{З}_{\text{соц}} =
    \frac {\text{З}_{\text{о}} + \text{З}_{\text{д}} \cdot \text{Н}_{\text{соц}}}
          {\num{100}} \text{\,,}
\end{equation}
\begin{explanation}
    где & $ \text{Н}_{\text{соц}} $ & норматив отчислений на социальные нужды \num{34}\% и обязательное страхование, \num{0,6} \% .
\end{explanation}

\begin{equation}
  \label{eq:econ:com_calc}
  \text{З}_{\text{д}} =
    \frac{(\num{102783,6} + \num{15417,52}) \times \num{34}}
         {\num{100}} = \num{40188,38}{\text{\byr}}
\end{equation}

Расчет прочих затрат осуществляется в процентах от затрат на основную заработную плату команды разработчиков с учетом премии по формуле:

\begin{equation}
  \label{eq:econ:etc}
  \text{З}_{\text{пз}} =
    \frac {\text{З}_{\text{о}} \cdot \text{Н}_{\text{пз}}}
          {\num{100}} \text{\,,}
\end{equation}
\begin{explanation}
    где  & $ \text{Н}_{\text{пз}} $ & норматив прочих затрат, \num{100} \% .
\end{explanation}

\begin{equation}
  \label{eq:econ:etc}
  \text{З}_{\text{пз}} =
    \frac{\num{102783,6} \times \num{100}}
         {\num{100}} = \num{102783,6}{\text{\byr}}
\end{equation}

Полная сумма затрат на разработку программного обеспечения находится путем суммирования всех рассчитанных статей затрат (\ref{table:econ:exspenses}).

\begin{table}[!ht]
\caption{Затраты на разработку программного обеспечения}
\label{table:econ:exspenses}
  \centering
  \begin{tabular}{| >{\raggedright}m{0.8\textwidth}
                  | >{\centering\arraybackslash}m{0.17\textwidth}|}
    \hline
    {\begin{center}
      Статья затрат
    \end{center} } & Сумма,\byr \\
    \hline
    Основная заработная плата команды разработчиков & \num{102783,6}\\

    \hline
    Дополнительная заработная плата команды разработчиков & \num{15417,52} \\

    \hline
    Отчисления на социальные нужды & \num{40188,38}  \\

    \hline
    Прочие затраты & \num{102783,6}  \\

    \hline
    Общая сумма затрат на разработку & \num{261173,1}  \\

    \hline

  \end{tabular}
\end{table}

\subsection{Оценка результата (эффекта) от продажи ПО}
\label{sub:econ:evaluation_result}

Экономический эффект представляет собой прирост чистой прибыли,
полученный организацией в результате использования разработанного ПО. Как правило, он может быть достигнут за счет:

\begin{itemize}
\item уменьшения (экономии) затрат на заработную плату за счет замены «ручных» операций и бизнес-процессов информационной системой;
\item ускорения скорости обслуживания клиентов и рост возможности обслуживания большего их количества в единицу времени, т.е. рост производительности труда;
\item появления нового канала сбыта продукции или получения заказов (как в случае внедрения интернет-магазина);
\item и т.п.
\end{itemize}

Экономический эффект организации-разработчика программного обеспечения в данном случае заключается в получении прибыли от его продажи множеству потребителей. Прибыль от реализации в данном случае напрямую зависит от объемов продаж, цены реализации и затрат на разработку данного ПО.

Таким образом, необходимо сделать обоснование предполагаемого объема продаж – количества копий (лицензий) программного обеспечения, которое будет куплено клиентами за год (N). Принимая во внимания примерное количество скачиваний аналогичных приложений в год (2 тыс. человек) и учитывая различия в количестве функций, можно спрогнозировать 200 скачиваний за год.

Далее следует определить цену на одну копию (лицензию) ПО. Цена формируется на рынке под воздействием спроса и предложения. Тогда расчет прибыли от продажи одной копии (лицензии) ПО осуществляется по формуле:

\begin{equation}
  \label{eq:econ:profit}
  \text{П}_{\text{ед}} = \text{Ц} - \text{НДС} -
    \frac {\text{З}_{\text{о}} \cdot \text{Н}_{\text{пз}}}
          {\num{100}} \text{\,,}
\end{equation}
\begin{explanation}
    где & $ \text{Ц} $ & цена реализации одной копии (лицензии) ПО, \byr; \\
        & $ \text{З}_{\text{р}} $ & сумма расходов на разработку и реализацию, \byr; \\
        & $ \text{N} $ & цена реализации одной копии (лицензии) ПО, \byr; \\
        & $ \text{П}_{\text{ед}} $ & прибыль, получаемая организацией-разработчиком от реализации одной копии программного продукта, \byr; \\
        & $ \text{НДС} $ & цена реализации одной копии (лицензии) ПО, \byr;
\end{explanation}

Цена одной копии (лицензии) программного обеспечения выбирается на основе цены на аналогичное программное обеспечение на рынке. Для приложений данного направления, средняя стоимость составляет 1500 руб.

Сумма налога на добавленную стоимость рассчитывается по формуле:

\begin{equation}
  \label{eq:econ:tax}
  \text{НДС} =
    \frac{\text{Ц} \cdot \text{НДС}}
         {\num{100} + \text{НДС}} \text{\,,}
\end{equation}
\begin{explanation}
    где & $ \text{НДС} $ & ставка налога на добавленную стоимость, \num{20} \%.
\end{explanation}

\begin{equation}
  \label{eq:econ:tax_cal}
  \text{НДС} =
    \frac{\num{200} \times \num{20}}
         {\num{20} + \num{100}} = \num{33,33}{\text{\byr}}
\end{equation}

Подставив значения в формулу 6.9, получим:

\begin{equation}
  \label{eq:econ:profit_calc}
  \text{П}_{\text{ед}} = \num{200} - \num{33,33} -
    \frac {\num{261173,1}}
          {\num{200}} = \num{160,8}{\text{\byr}} 
\end{equation}

Суммарная годовая прибыль по проекту в целом будет равна:

\begin{equation}
  \label{eq:econ:year_income}
  \text{П} =
    \text{П}_{\text{ед}} \cdot \text{N} \text{\,.}
\end{equation}

\begin{equation}
  \label{eq:econ:year_income_calc}
  \text{П} =
    \num{160,8} \times \num{200} = \num{32160} \text{\byr}
\end{equation}

Рентабельность затрат на разработку ПО рассчитывается по следующей формуле:

\begin{equation}
  \label{eq:econ:profitability}
  \text{Р} = \num{100} \times
    \frac {\num{32160}}
          {\num{261173,1}}\text{\byr}
\end{equation}

Проект является экономически эффективным, т.к. рентабельность затрат на разработку программного обеспечения превышает среднюю процентную ставку по банковским депозитным вкладам.

Чистая прибыль рассчитывается по формуле:
\begin{equation}
  \label{eq:econ:net_profit}
  \text{ЧП} = \text{П} -
    \frac {\text{П} \cdot \text{Н}_{\text{приб}}}
          {\num{100}} \text{\,,}
\end{equation}
\begin{explanation}
    где  & $ \text{Н}_{\text{приб}} $ & ставка налога на прибыль, \num{18} \% .
\end{explanation}

\begin{equation}
  \label{eq:econ:net_profit_calc}
  \text{ЧП} = \num{32160} -
    \frac {\num{32160} \cdot \num{18}}
          {\num{100}} = \num{26371,2} \text{\byr}
\end{equation}


\subsection{Расчёт показателей эффективности инвестиций в разработку ПО}
\label{sub:econ:investment_efficiency}

Подставив значения, получим:

Перед тем, как произвести расчёт показателей эффективности инвестиций в разработку ПО, необходимо сравнить размер инвестиций в разработку и получаемый годовой экономический эффект. Т.к. сумма инвестиций меньше, чем сумма годового эффекта, инвестиции окупятся менее чем за год. В таком случае, необходимо рассчитать лишь рентабельность инвестиций:

\begin{equation}
  \label{eq:econ:net_profit}
  \text{Р}_{\text{и}} =
    \frac {\text{П}_{\text{ч}}}
          {\text{C}_{\text{п}}}
           \cdot 100\% \text{\,,}
\end{equation}

Подставив значения, получим:

\begin{equation}
  \label{eq:econ:net_profit_calc}
  \text{Р}_{\text{и}} =
    \frac {\num{26371,2}}
          {\num{261173,1}}
           \cdot 100\% = \num{10,10} \%.
\end{equation}


