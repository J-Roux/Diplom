\sectioncentered*{ЗАКЛЮЧЕНИЕ}
\addcontentsline{toc}{section}{ЗАКЛЮЧЕНИЕ}
\label{sec:outro}

В результате дипломного проектирования были проведены исследования в области выделения информационных образов и сервисов рекомендации. Было проведено исследование существующих способов выделения информационных признаков при решении задач: распознования неперкуссионных инструментов, распознования заимствований и классификации жанров. Были рассмотрены существующие сервисы и приложения рекомендации музыки на основе акустического анализа, рассмотрены их  положительные и отрицательные стороны. 
На основе исследования были выбраны временные, спектральные, ритмические(перкуссионные) признаки,  а также мел-кепстаральные коэффициенты. 

Был спроектирован и реализован программный модуль по выделению выбранных признаков. Выделенные признаки были использованы для решения задачи жанровой классификации для 10 жанров музыки с помощью различных методов классификации. Было проведено сравнение эффективности методов классификации как отдельных фрагментов, так и целого трека. Решение задачи  жанровой классификации показало, что выбранные признаки значимы и на их основе можно делать рекомендацию.  Также на были выделены жанры, которые лучше всего выделяются всеми методами классификации. Для этих жанров была сделана визуализации в двумерном и трёхмерном измерении путйм уменьшение размерености векторов признаков алгоритмами t-SNE и PCA.

 