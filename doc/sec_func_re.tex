\section{ФУНКЦИОНАЛЬНОЕ ПРОЕКТИРОВАНИЕ}
\label{sec:func}


\subsection{Алгоритм работы модуля выделения информационных образов из музыкального произведения}
    
На ход модуля чтения музыкального произведения подаётся путь к музыкальному произведению. Файл с расширением “.wav” считывается в оперативную память в формате массива отсчётов. Файл с иным расширением конвертируются в WAV формат. 

В модуле препроцессинга и нарезки от музыкального трека отрезается 10\% длины с начала и конца трека. В случае, если трек является многоканальным, то он приводится к одноканальному посредством чередованием правого и левого канала. Данные нормализуются по МО и СКО. Затем используется экспоненциальное сглаживание с коэффициентом сглаживания 0,99. Результат нарезается на фрагменты по 5 секунд с перекрытием в 0,5 секунды.

В модуле получения частотно-временного представления используется оконное преобразование Фурье с окном Хэмминга шириной в 10 миллисекунд. Также в модуле для получения ритмического образа используется вейвлет Добеши каскадным алгоритмом с количеством каскадов 4. Результат каждого каскада сглаживается экспоненциальным сглаживанием с коэффициентом сглаживания 0,97, передискретизируется с уменьшением дискретизации в 16 раз, нормализуется по МО. Затем результаты поэлементно складываются и вычисляется автокорелляция. Результатом работы модуля является спектрограмма и коррелограмма.

В модуле выделения информационных образов из временного представления сигнала выделяется временной, из спектрограммы - спектральный образ по каждому срезу спектрограммы, из коррелограммы - ритмический. Мел-кепстральные коэффициенты считаются по окнам в 5 миллисекунд. 

В модуле обработки информационных образов по всем образам кроме временного считается МО, СКО, коэффициент ассиметрии, коэффициент эксцесса.

В базу данных записывается информационный образ в формате: название произведения, номер фрагмента и результат работы модуля обработки информационных образов.  

В модуле классификации информационных образов используются набор стандартных алгоритмов классификации из библиотеки scikit-learn с параметрами по умолчанию:
\begin{enumerate}
\item метод ближайших соседей с k = 3;
\item метод опорных векторов с полиномиальным ядром;
\item дерево принятия решений;
\item случайный лес;
\item нейронная сеть прямого распространения с 1000 нейронов на скрытом слое;
\item adaBoost;
\itemнаивный баесовский классификатор;
\item квадратичный дискриминант.
\end{enumerate}

Для оценки качества классификации использовался скользящий контроль количеством разбиений равным 10 и алгоритмом разбиения stratified k-fold. Также для каждого классификатора строится матрица ошибок.

В модуле визуализации используется алгоритм t-sne.Это нелинейный метод уменьшения размерности, который особенно хорошо подходит для вложения высокоразмерных данных в пространство двух или трех измерений, которое затем можно визуализировать на диаграмме рассеяния. В частности, он моделирует каждый высокоразмерный объект с помощью двух- или трехмерной точки таким образом, что аналогичные объекты моделируются соседними точками, а разнородные объекты моделируются удаленными точками. Данные визуализируются в 2д и 3д графики. 


\subsection{Реализация внутренних модулей}
Рассмотрим в деталях функциональные части системы: для этого произведем детальный анализ компонентов, модулей, составляющих их классов и отдельных методов, реализующих логику программы.



Структурно система подразделяется на ряд модулей , объединенных под общим модулем с названием \texttt{music\_feature\_extractor}. Дальнейшее разделение ведется по следующим модулям:
\begin{enumerate}[label=\arabic*.]
\item{Модуль чтения музыкального произведения - \texttt{WavModule}.}
\item{Модуль препроцессинга и нарезки - \texttt{PreprocessingModule}.}
\item{Модуль получения частотно-временного представления сигнала -
 
\texttt{SpectralTransformerModule}.}
\item{Модуль извлечения информационных образов -

\texttt{FeatureExtractorModule}.}
\item{Модуль обработки информационных образов - 

\texttt{FeatureProcessingModule}. }
\item{База данных информационных образов - \texttt{DatabaseModule}.}
\item{Модуль жанровой классификации музыкального произведения -

\texttt{GenreClassificationModule}.}
\item{Модуль визуализации - \texttt{VisualizeDataModule}.}
\end{enumerate}



